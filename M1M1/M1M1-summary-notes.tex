%using document class from KOMA-Script
\documentclass{scrartcl}
\title{M1M1 Summary Notes}
\subtitle{JMC Year 1, 2017/2018 syllabus}
\date{}
\author{Fawaz Shah}

% Packages for adding hyperlinks to table of contents
\usepackage{color}   %May be necessary if you want to color links
\usepackage{hyperref}
\hypersetup{
    colorlinks=true, %set true if you want colored links
    linktoc=all,     %set to all if you want both sections and subsections linked
    linkcolor=black,  %choose some color if you want links to stand out
}

%package that allows aligned equations
\usepackage{amsmath}

%package that allows notation for extra mathematical symbols
\usepackage{amssymb}

%new commands for popular sets for ease of use
\newcommand{\R}{\mathbb{R}}
\newcommand{\N}{\mathbb{N}}
\newcommand{\Z}{\mathbb{Z}}
\newcommand{\Q}{\mathbb{Q}}
\newcommand{\C}{\mathbb{C}}

% renaming command to writing vectors in bold notation
\renewcommand{\vec}[1]{\mathbf{#1}}

%package for managing images
\usepackage{graphicx}
\graphicspath{ {../img/} }

%package for managing hyperlinks
\usepackage{hyperref}

%package for added blue colored boxes
\usepackage[most]{tcolorbox}

%package that allowed removing indent from enumerate environment
\usepackage{enumitem}

%package that allows for negating nearly any symbol
\usepackage{centernot}

%enlarging line spacing in tables
\renewcommand{\arraystretch}{2}

\begin{document}
\large
\maketitle
\noindent This document contains a bunch of definitions and techniques, sorted by category.
\\\\
M1M1 is more about applying mathematical methods rather than proving theorems, so sly manipulation of mathematical techniques is crucial. Most of the content is A Level, so I have only included relatively non-A-Level content.
\\\\
Boxes cover content in more detail.
\tableofcontents
\newpage

\section{Trigonometric identities}
Several useful trig identities (on top of the ones from A Level):
\begin{align}
\sin(x + \frac{\pi}{2}) & = \cos(x) \\
\cos(x + \frac{\pi}{2}) & = \sin(x) \\
\\
\sin(x + \pi) & = - \sin(x) \\
\cos(x + \pi) & = - \cos(x) \\
\\
\sin(x + 2 \pi) & = \sin(x) \\
\cos(x + 2 \pi) & = \cos(x) \\
\end{align}

\section{Even and odd functions}
An even function is such that, for all $ x $:
\begin{equation}
f(x) = f(-x)
\end{equation}
An odd function is such that, for all $ x $:
\begin{equation}
f(x) = -f(-x)
\end{equation}
Any function can be expressed as the sum of an even and an odd function. We can write any function as follows:
\begin{equation}
f(x) = \frac{1}{2}(f(x) + f(-x)) + \frac{1}{2}(f(x) - f(-x))
\end{equation}
The left part of the function is even, the right part is odd.

\section{Power series}
The general form of a power series is:
\begin{equation}
\sum_{n=0}^{\infty} a_{n} x^{n}
\end{equation}
for some sequence $ a_{n} $.
\\\\ 
There are several power series which must be memorized:
\begin{align}
e^{x} & = 1 + x + \frac{x^{2}}{2!} + \frac{x^{3}}{3!} + ... \\
\sin(x) & = x - \frac{x^{3}}{3!} + \frac{x^{5}}{5!} - ... \\
\cos(x) & = 1 - \frac{x^{2}}{2!} + \frac{x^{4}}{4!} - ... \\
\sinh(x) & = x + \frac{x^{3}}{3!} + \frac{x^{5}}{5!} + ... \\
\cosh(x) & = 1 + \frac{x^{2}}{2!} + \frac{x^{4}}{4!} + ... \\
\arctan(x) & = x - \frac{x^{3}}{3} + \frac{x^{5}}{5} - ... \quad |x| < 1 \\
\ln{(1+x)} & = x - \frac{x^{2}}{2} + \frac{x^{3}}{3} - ... \quad |x| < 1 \\
\frac{1}{1-x} & = 1 + x + x^{2} + x^{3} + ... \quad |x| < 1 \\
\frac{1}{1+x} & = 1 - x + x^{2} - x^{3} + ... \quad |x| < 1 \\
(1+x)^{n} & = 1 + nx + \frac{n(n-1)}{2!}x^{2} + ... \quad |x| < 1
\end{align}
Every power series has a corresponding radius of convergence $ R $, where the series converges for $ |x| < R $ and diverges otherwise. This can be found using the ratio test (see section \ref{series}).
\\\\
If a power series converges for all $ x $, then we say $ R = \infty $.
\\\\
Power series can be added, subtracted, multiplied, divided, differentiated, integrated and also substituted into one another within their radius of convergence.

\subsection{Maclaurin vs. Taylor series}
The general form of a Maclaurin series is:
\begin{equation}
f(x) = f(0) + f'(0)x + \frac{f''(0)}{2!}x^{2} + ...
\end{equation}
The general form for a Taylor series is:
\begin{equation}
f(x) = f(a) + f'(a)(x-a) + \frac{f''(a)}{2!}(x - a)^{2} + ...
\end{equation}
So a Maclaurin series is a special case of a Taylor series, just centred about $ x = 0 $.
\\\\
The entire point of power series in general is that we can approximate the value of a function at any point $ x $, given we know all the derivatives at a certain point $ a $. Realistically $ x $ must be close to $ a $.
\\\\
The Taylor series for a polynomial is finite, therefore we can calulate the actual values of polynomial functions. The Taylor series for non-polynomials are infinite series, so we must truncate the series at some point to calculate a value. This is why this method only gives an approximation to the actual value.

\subsection{Complex power series} \label{complexpowerseries}

We have to know how to calculate the radius of convergence for a complex power series, i.e. a series of the form:
\begin{equation}
\sum_{i=0}^{n} c_{n} z^{n}
\end{equation}
This involves a result from $ 2^{\textrm{nd}} $ year maths. Note that the radius of convergence of a complex power series is the radius of a disk in the Argand diagram, centered about $ (0, 0) $. If a complex function has no singularities (i.e. nowhere where it 'blows up') in a disk of radius $ R $ centered about $ \omega \in \C $, then it converges absolutely for $ |z - \omega| < R $.
\\\\
Therefore the radius of convergence about any point $ \omega \in \C $ is the distance between $ \omega $ and the nearest singularity.

\section{Limits and continuity}
Here we provide the $ \epsilon - \delta $ definition of a limit $ L $ at $ x = a $:
\begin{equation}
\forall \epsilon > 0 \quad \exists \delta > 0 \ \textrm{ s.t. } \ |x - a| < \delta \Rightarrow |f(x) - L| < \epsilon
\end{equation}
The limit at $ a $ only exists if:
\begin{equation}
\lim_{x \to a^{-}} f(x) = \lim_{x \to a^{+}} f(x)
\end{equation}
\\
\textit{Properties of limits}
\begin{align}
\lim_{x \to a^{}} (f(x) + g(x)) & = \lim_{x \to a^{}} f(x) + \lim_{x \to a^{}} g(x) \\
\lim_{x \to a^{}} (f(x) \ g(x)) & = \lim_{x \to a^{}} f(x) \lim_{x \to a^{}} g(x)
\end{align}
\\
\textit{L'H\^{o}pital's Rule}
\\
If we have an indeterminate limit (that is, a limit that evaluates to $ \frac{0}{0} $ or $ \frac{\infty}{\infty} $) then we can employ L'H\^{o}pital's Rule:
\begin{equation}
\lim_{x \to a^{}} \frac{f(x)}{g(x)} = \lim_{x \to a^{}} \frac{f'(x)}{g'(x)}
\end{equation}
\\\\
A function is continuous at $ a $ if:
\begin{equation}
\lim_{x \to a} f(x) = f(a)
\end{equation}
If the limit of a function $ f $ does not exist at $ a $, then $ f $ is not continuous at $ a $ (note that the opposite is NOT necessarily true).

\section{Differentiation}
Derivatives of the main trigonometric and hyperbolic functions should be derivable. These are:
\begin{center}
\begin{tabular}{c c c}
sin & cos & tan \\
cosec & sec & cot \\
arcsin & arccos & arctan \\
\end{tabular}
\end{center}

\begin{center}
\begin{tabular}{c c c}
sinh & cosh & tanh \\
cosech & sech & coth \\
arcsinh & arccosh & arctanh \\
\end{tabular}
\end{center}
\textit{Leibniz' Rule}
\\
To take the $ n^{th} $ derivative of a product of functions $ f(x) \cdot g(x) $, we can use Leibniz' rule:
\begin{equation}
\frac{d^{n}}{d x^{n}} (f(x) \cdot g(x)) = \sum_{r=0}^{n} {n \choose r} f^{(r)}(x) \cdot g^{(n-r)}(x)
\end{equation}

\section{Conic sections}
There are 3 types of conic sections:
\begin{itemize}
\item ellipses
\item hyperbolas
\item parabolas
\end{itemize}
Any conic section can be expressed in polar form as:
\begin{equation}
r = \frac{L}{1 + e \cos(\theta)}
\end{equation}
where $ e $ varies per type. The main differences are illustrated in the following table:
\begin{center}
\begin{tabular}{|c|c|c|}
\hline
\textbf{conic section} & \textbf{cartesian form} & \textbf{value of $ e $ for polar form} \\
\hline
ellipse & $ \frac{x^{2}}{a^{2}} + \frac{y^{2}}{b^{2}} = 1 $ & $ e < 1 $ \\
\hline
parabola & $ y = ax^{2} $ & $ e = 1 $ \\
\hline
hyperbola & $ \frac{x^{2}}{a^{2}} - \frac{y^{2}}{b^{2}} = 1 $ & $ e > 1 $ \\
\hline
\end{tabular}
\end{center}
Special case: if $ e = 0 $ in polar form, we get a circle.

\section{Series} \label{series}
A series is the sum of elements of a sequence $ a_{n} $. We say the series is convergent (i.e. $ a_{n} $ is 'summable') if the sum to infinity exists. The sum to infinity is:
\begin{equation}
\sum_{i=0}^{\infty} a_{i} = \lim_{n \to \infty} \sum_{i=0}^{n} a_{i}
\end{equation}
We say a series $ \sum_{i=0}^{n} a_{i} $ is absolutely convergent if $ \sum_{i=0}^{n} |a_{i}| $ is convergent.
\\\\
A series that is convergent but not absolutely convergent is called conditionally convergent.
\\\\
To evaluate whether a series is convergent or divergent, we can use some standard tests.
\\\\
\\
\textit{Preliminary test}
\\
If $ a_{n} \centernot\to 0 $ then $ \sum_{i=0}^{n} a_{n} $ diverges.
\\\\
\textit{Alternating series test}
\\
If $ a_{n} $ is alternating, and $ |a_{n}| $ is decreasing, and $ a_{n} \to 0 $, then 
$ \sum_{i=0}^{n} a_{n} $ converges.
\\\\
\textit{Comparison test}
\\
If $ |a_{n}| < |b_{n}| $ for all $ n $:
\begin{itemize}
\item If $ \sum_{i=0}^{n} b_{i} $ converges absolutely, then so does $ \sum_{i=0}^{n} a_{i} $
\item If $ \sum_{i=0}^{n} a_{i} $ diverges, then so does $ \sum_{i=0}^{n} b_{i} $
\end{itemize}
\noindent
\\
\textit{Integral test}
\\
If $ a_{n} $ is $ +ve $ and decreasing, and:
\begin{equation}
\int_{N}^{\infty} a_{n} \ dn
\end{equation}
exists for some $ N \in \N $, then $ a_{n} $ converges.
\\\\
\textit{Ratio test}
\\
Consider the following limit:
\begin{equation}
L = \lim_{n \to \infty} |\frac{a_{n+1}}{a_{n}}|
\end{equation}
\begin{itemize}
\item If $ L < 1 $, the series is absolutely convergent (in particular the sequence itself converges to $ 0 $).
\item If $ L > 1 $ the series diverges.
\item $ L = 1 $ is an indeterminate case.
\end{itemize}
\noindent
\\
If a series is absolutely convergent then the convergent sum does not depend on the order of summation.
\\\\
If a series is conditionally convergent then the summed terms can be rearranged to produce any convergent sum you want.

\section{Complex numbers}
Mostly the same as M1F. See M1F notes.
\\\\
Some useful identites:
\begin{align}
\cos(z) & = \cosh(iz) \\
\sin(z) & = i \sinh(iz)
\end{align}
\\
For calculating the radius of convergence of a complex power series, see section \ref{complexpowerseries}.

\section{Fundamental Theorem of Calculus}
The fundamental theorem of calculus defines the antiderivative $ F(x) $ and shows the relation between differentiation and integration. It can be expressed in several different forms:
\begin{equation} \label{ftcalculus1}
\frac{d}{dx} \int_{a}^{x} f(t) \ dt = f(x)
\end{equation}
\begin{equation}
\int_{a}^{b} f(x) \ dx = F(b) - F(a)
\end{equation}
Note that in equation \ref{ftcalculus1}, any lower bound $ a $ and dummy variable $ t $ can be picked, without affecting the validity of the theorem.

\section{Integrating functions of two variables}
Similar to how a single integral represents the \textbf{area} underneath a 2D curve $ y = f(x) $, a double integral represents the \textbf{volume} underneath a 3D curve $ z = f(x, y) $. The double integral over a region $ S $ is denoted as:
\begin{equation}
\iint_{S} f(x, y) \ dx \ dy
\end{equation}
If we say $ S $ ranges from say, $ x = a $ to $ x = b $ and from $ y = c $ to $ y = d $, we can write the double integral as:
\begin{equation}
\int_{c}^{d} \int_{a}^{b} f(x, y) \ dx \ dy
\end{equation}
To evaluate the double integral, simply evaluate the inner integral w.r.t $ x $ and then integrate the result w.r.t $ y $.
\\\\
\textit{Changing into polar form}
\\
Suppose in the region $ S $, $ f(x, y) = g(r, \theta) $ for some function $ g $, where $ 0 \leq \theta \leq 2 \pi $ and $ r = h(\theta) $ for some function $ h $. We can convert the double integral into polar coordinates (this might perhaps make it easier to evaluate). HOWEVER, we must put an extra $ r $ term in the part of the integral w.r.t $ r $. This is extra bit is called the \textbf{Jacobian} (not covered fully in M1M1).
\begin{equation}
\iint_{S} f(x, y) \ dx \ dy = \int_{0}^{2 \pi} \int_{0}^{h(\theta)} g(r, \theta) \cdot r \ dr \ d \theta
\end{equation}

\section{Lengths, areas and volumes}
The length of a curve $ y = f(x) $ between $ x = a $ and $ x = b $ is given by:
\begin{equation}
L = \int_{a}^{b} \sqrt{1+(f'(x))^{2}} \ dx
\end{equation}
For parametric curves defined by $ x(t) $ and $ y(t) $, the length is given by:
\begin{equation}
L = \int_{t_{a}}^{t_{b}} \sqrt{(x'(t))^{2}+(y'(t))^{2}} \ dt
\end{equation}
For polar curves defined by $ r = f(\theta) $, the length is given by:
\begin{equation}
L = \int_{\theta_{a}}^{\theta_{b}} \sqrt{(f(\theta))^{2}+(f'(\theta))^{2}} \ d \theta
\end{equation}
The surface area of revolution obtained by rotating a curve $ f(x) $ about the $ x $ axis is given by:
\begin{equation}
A = 2 \pi \int_{a}^{b} f(x) \sqrt{1+(f'(x))^{2}} \ dx
\end{equation}
The volume of revolution obtained by rotating a curve $ f(x) $ about the $ x $ axis is given by:
\begin{equation}
V = \pi \int_{a}^{b} (f(x))^{2} \ dx
\end{equation}

\section{Ordinary Differential Equations (ODEs)}

\subsection{1st order linear ODEs}
Every 1st order linear ODE can be expressed as:
\begin{equation}
\frac{dy}{dx} + p(x)y = q(x)
\end{equation}
These can ALL be solved by the \emph{integrating factor} method:
\begin{enumerate}
\item Multiply both sides by $ exp(\int_{}^{} p(x) \ dx) $
\item Use the reverse product rule to express the LHS as a single derivative (of a function of y).
\item Integrate both sides and rearrange.
\end{enumerate}

\subsection{Separable ODEs}
Separable equations can be written in the form:
\begin{equation}
\frac{dy}{dx} = f(x)g(y)
\end{equation}
These can be rearranged and integrated on both sides, with respect to the different variables.

\subsection{Homogenous ODEs}
Homogenous equations can be written in the form:
\begin{equation}
\frac{dy}{dx} = f(\frac{y}{x})
\end{equation}
To solve, set $ v = \frac{y}{x} $, so that $y = xv $. Note that v is still a single-variable function of x, since y is a function of x.
Now we can differentiate both sides to get:
\begin{equation}
\frac{dy}{dx} = v + x\frac{dv}{dx}
\end{equation}
We now have simultaneous equations for $ \frac{dy}{dx} $. Equate and solve for $ \frac{dv}{dx} $, and then solve this 1st order linear ODE in $ \frac{dv}{dx} $ to find v (and then y).

\subsection{Bernoulli type ODEs}
A Bernoulli type ODE is of the form:
\begin{equation}
\frac{dy}{dx} + p(x)y = q(x)y^n
\end{equation}
To solve:
\begin{enumerate}
\item Multiply both sides by $ (1-n)y^{-n} $
\item Let $ z = y^{1-n} $ and substitute into equation, including rewriting one of the terms as $ \frac{dz}{dx} $
\item The resulting equation is 1st order linear in z, so solve for z (and then y).
\end{enumerate}

\end{document}